\hypertarget{index_purpose}{}\section{Purpose}\label{index_purpose}
The Encoder class enables the user to implement multiple encoders without needing to rewrite code for each one. All encoders can run and be read from independently. All encoders can be zeroed independently.\hypertarget{index_usage}{}\section{Usage}\label{index_usage}
An Encoder object is constructed by providing the pin objects, timer ID, and timer channels corresponding to the pins to which the physical encoder is attached. Best results occur when the read() function is called often. The encoder can be zeroed at any position chosen by the user. Please see \mbox{\hyperlink{classencoder_1_1_encoder}{encoder.\+Encoder}} which is part of the encoder package.\hypertarget{index_testing}{}\section{Testing}\label{index_testing}
The Encoder class was tested manually with a physical encoder. The read() function was called at a high frequency (100 Hz).\hypertarget{index_bugs_and_limitations}{}\section{Bugs and Limitations}\label{index_bugs_and_limitations}
The Encoder class handles timer counter underflow/overflow correctly only when the read() function is called at a high enough frequency, determined as a function of the speed at which the motor spins. Best results are produced when the read() function is called as frequently as reasonable.\hypertarget{index_location}{}\section{Location}\label{index_location}
\href{https://github.com/rahul-go/Mechatronics-LAB/tree/master/Lab%201}{\tt https\+://github.\+com/rahul-\/go/\+Mechatronics-\/\+L\+A\+B/tree/master/\+Lab\%201}

\begin{DoxyAuthor}{Author}
Rahul Goyal, Cameron Kao, Harry Whinnery
\end{DoxyAuthor}
\begin{DoxyCopyright}{Copyright}
License Info
\end{DoxyCopyright}
\begin{DoxyDate}{Date}
January 24, 2019 
\end{DoxyDate}
